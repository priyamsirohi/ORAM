\section{Model}
\label{model}
\medskip
{\bf Deployment.~} \sysname~ considers a deployment model with two parties: the ORAM clients (with limited 
local storage) and the ORAM server (a remote storage that hosts the clients' data).
The server stores data as fixed sized ``blocks''. \sysname~ considers $N$ 
blocks of outsourced data on the server. Clients also access data in blocks 
addressed by a logical block ID denoted by $id$. 
The logical address space for all blocks is shared by the clients.
The parties engage in an interactive query-response based protocol established 
by \sysname~. The communication channel between the clients and the server is considered 
secure using SSL. 

{\bf Clients.~} Clients are considered honest in the \sysname~ model and 
do not interact with each other. Further, the clients share the key to the ORAMs and 
the secret hash functions used for PD-ORAM, which are stored encrypted on the server. 
Clients can engage the server without having any knowledge of other client states. 
Any locking mechanism (as required by the protocol) is imposed by the server. 
\sysname~ does not consider the case of malicious clients.

{\bf Server.~} \sysname~ considers an untrusted server that is honest but curious and 
does not deviate from the \sysname~ protocol. The server can observe  
all requests and try to correlate them by saving and comparing snapshots (state of the 
ORAMs after each query). It stores the ORAM keys, hash functions and other access counters as 
required by the \sysname~ protocol. Further, the server maintains and duly increments the counters. 
\sysname~ does not consider a malicious server than can mount replay attacks and ``fork'' client views.


{\bf Security challenge.~} Any system that supports multi-client concurrent access for ORAMs needs to prevent two possible security leaks -

\begin{itemize}
 \item Correlation between data items concurrently accessed in a single round.
 \item Correlation between data items accessed in successive rounds of one or more concurrent accesses.
\end{itemize}

In the above context, we define multi-client concurrent access security for ORAMs as a security game, where the challenger is 
a fixed set of clients, $\mathcal{C} = \{ c_1, c_2, \ldots c_n\}$, and the adversary, $\mathcal{A}$ is the remote server that hosts a database 
uploaded by $\mathcal{C}$. All items in the database are indexed and can be accessed concurrently by the clients. 


\begin{enumerate}
\item $\mathcal{A}$ and $\mathcal{C}$ engage in polynomial rounds of the following query-response based protocol
\begin{enumerate}
  \item $\mathcal{A}$ selects two sets of item indices $\mathcal{O}_{1} = \{x_1,x_2, \ldots,x_n\}$ and $\mathcal{O}_{2} = \{y_1,y_2, \ldots,y_n\}$ such that 
 $x_i \neq x_j$ and $y_i \neq y_j \forall i,j \in [1,n]$. 
 $\mathcal{A}$ sends $\mathcal{O}_{1}(i)$ and $\mathcal{O}_{2}(i)$ to $c_{i}$ where $\mathcal{O}_{j}(i)$ is the $i^{th}$ item in $\mathcal{O}_{j}$. 
 \item On the basis of a fairly selected bit $b$, the clients query for the items in $\mathcal{O}_{b}$.
 \item Observing the queries in Step 2, $\mathcal{A}$ outputs bit $b^{'}$
 \item $\mathcal{A}$ wins the round iff. $b^{'} = b$. 
 \end{enumerate}
 \item $\mathcal{A}$ wins the security game iff. she can win any round with non-negligible advantage over random guessing where non-negligibility is defined 
 over an implementation specific security parameter.
 \end{enumerate}

A mechansim that satisfies the above security game ensures that it protects against the two aforementioned leaks. To see why consider the following: 

{\bf Correlation between data items accessed in a single round.~}
%
The security game straightforwardly ensures that $\mathcal{A}$ can win a round with non-negligible advantage over guessing if she can 
distinguish between two sequence of concurrent accesses. Therefore, a mechanism that satifies the game ensures that an adversary cannot correlate 
items concurrently accessed in the same round. 

{\bf Correlation between data items accessed in successive rounds.~}
Consider two randomly chosen sets of item $\mathcal{O}_{1}$, $\mathcal{O}_{2}$ and $\mathcal{O}_{1}$. In two successive rounds, $\mathcal{A}$ provides the following 
sets of items to $\mathcal{C}$:

\begin{itemize}
 \item Round1: $\mathcal{O}_{1}$ and $\mathcal{O}_{2}$
 \item Round2: $\mathcal{O}_{1}$ and $\mathcal{O}_{3}$
 \end{itemize}

In this case, if $\mathcal{C}$ queries for $\mathcal{O}_{1}$ both in round 1 and 2, and $\mathcal{A}$ could correlate accesses in the current round 
with previous rounds, $\mathcal{A}$ can win round 2 with non-negligible advantage. Therefore, a mechanism that satisfies the security game also ensures that 
an adversary cannot correlate accesses in successive rounds.

\sysname~ gaurantees multi-client concurrent access security in context of the security game by ensuring access pattern privacy of concurrent accesses 
in each round using PD-ORAM and RingORAM as described in Section \ref{oram:overview}.