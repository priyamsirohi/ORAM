\section{Related}
\label{oram:related}

ORAMs have been well-researched since the seminal work by Goldreich and Ostroevsky \cite{goldreich}. 
The construction provided by Goldreich and Ostroevsky requires only logarithmic storage at the client 
side and has an amortized communication complexity of $\order{log^{3}N}$. ORAM constructions can be 
broadly classified into two categories: pyramid based construction and tree based constructions. 
Below we briefly discuss these constructions.


\subsection{Pyramid based ORAM}
\label{oram:related:pyramid}
%
The first pyramid based construction was provided by Goldreich and 
Ostroevsky \cite{goldreich}. A pyramid based construction organized the data in $logN$ levels. 
Level $i$ consists of $4^{i}$ data blocks assigned to one of $4^{i}$ buckets as determined by a secure 
hash function. Due to hash collisions, each bucket can contain up to $\order{ln N}$ blocks.

{\bf Read.~}
%
To obtain a particular block, a client scans {\em one} bucket from each level as determined by 
the hash of the logical block ID of that block. The ORAM maintains two invariants: i) a client 
access never reveals the level at which a block has been found, ii) a block is accessed 
from a particular location only once. To ensure (i), after a client has found the required block 
at a particular level, the client continues scanning a {\em random} bucket from all the levels 
below it. Once all the levels have been queried, the client reencrypts the accessed block and places 
it in the top level. This maintains (ii) since it ensures that a block that has been accessed earlier 
will be located in the top level for the next query for the same block. The rest of the search pattern will 
be random. 

{\bf Write.~} 
%
Writes proceed in exactly the same way as the reads with the exception that the updated value of the accessed block 
is placed in the top level. The same data access pattern for both reads and writes and semantic encryption ensure that 
the server cannot learn the purpose (read or write) of an access.

{\bf Overflow.~}
%
The top level overflows after a fixed number of accesses since accessed blocks are always placed there. In this case, the top level 
is merged with the second level and reshuffled using a new secure hash function. The reshuffling is done obliviously using a sorting 
network \cite{goldreich,bforam,randomizedshellsort}. Each level overflows once after the level above has overflown twice. To ensure invariant 
(i), all buckets contain the same number of blocks (either real or dummy) after a reshuffle.


The most expensive step of the pyramid ORAM is the reshuffle. Various mechanisms have been proposed to make the reshuffle 
more efficient. Williams and Sion \cite{usablepir} show how to achieve an amortized construction with $\order{log^{2}N}$ communication 
complexity under $\order{\sqrt{N}}$ client storage using an oblivious merge sort. Williams {\em et al.} further use this mechanism to 
build an ORAM with $\order{logN}$ access complexity and $\order{log^{2}N}$ overall communication complexity by storing an encrypted bloom 
filter on the server and retrieving one block from a level. 

Pinkas {\em et al.} \cite{Pinkasoram} use {\em cuckoo hashing} and randomized shell sort {\em randomizedshellsort} 
over the original Goldreich and Ostroevsky solution \cite{goldreich} and achieve an amortized communication 
complexity of $\order{log^{2}N}$ with constant client side storage. However, Goodrich {\em et al.} \cite{goodrichoram} 
highlight a leak in the construction in \cite{Pinkasoram} and provide an alternate construction thats achieves an amortized 
communication complexity of $\order{logN}$ under the assumption of $\order{N^{1/r}}$ client storage with $r > 1$. 

Although, the above mentioned constructions improve upon the original solution in \cite{goldreich}, client queries still 
need to wait for the duration of a reshuffle. De-amortized constructions allow queries and reshuffles to proceed together and 
thus eliminate clients waiting for reshuffles after a level overflow. Goodrich {\em et al.} \cite{goodrich_deamortized} 
show how to de-amortize the original square root solution and hierarchical solution \cite{goldreich} and achieve a worst-case 
complexity of $\order{logN})$ in the presence of $\order{n^{r}}$ client side storage where $r > 0$. In \cite{kushilevitzoram}, 
Kushilevitz {\em et al.} use cuckoo hashing and rotating buffers to provide a de-amortized construction of the original hierarchical 
solution \cite{goldreich} which achieves a worst-case communication complexity of $\order{log^{2}N/loglogN}$. 

{\bf PD-ORAM:} Unlike the de-amortization techniques used in \cite{kushilevitzoram,goodrich_deamortized} where each query performs an 
additional fixed amount of work for the reshuffle, the PD-ORAM \cite{privatefs} de-amortization abstraction performs a reshuffle 
in the background while monitoring progress to ensure that a level is ready after a reshuffle as soon as it is required. 
Further, queries can proceed simultaneously through a read-only variant of the level while the reshuffle takes place and ensures 
roughly similar query costs.

\subsection{Tree-based ORAM}
\label{oram:related:tree}
%
In contrast to de-amortized ORAM constructions, tree-based ORAMs are naturally un-amortized (the worst-case query cost is equal to the average cost). 
A tree based ORAM organizes the data as a binary (or ternary) tree. Each node of the tree is a bucket which can contain multiple blocks. 
A block is randomly mapped to a leaf in the tree. The ORAM maintains the following invariant: a block resides in any one of the buckets on the 
path from the root to the leaf to which the block is mapped. The position map which maps blocks to leaves is either stored on the client ($\order{N}$ 
storage required) or recursively on the server for $\order(1)$ client side storage at the cost of $\order{log^{2}N}$ increase in communication 
complexity.

To access a particular block, the client downloads all the buckets (or one element from each bucket) along the path from 
the root to the leaf to which the block is mapped. Once the block has been read, it is remapped to a new leaf and {\em evicted} 
back to the tree. Various eviction procedures have been proposed in literature \cite{circuitoram,pathoram,ringoram,binarytreeoram,clporam}

{\bf Binary Tree ORAM:} The original tree-based ORAM proposed by Shi {\em et al.} \cite{binarytreeoram} 
places the accessed block at the root of the tree and 
evicts a constant number of blocks from nodes at a particular depth to children nodes. In this case, the buckets need to be at least sized 
$\order{logN}$ and the overall access complexity for the construction is $\order{log^{3}N}$.

{\bf PathORAM:} In PathORAM \cite{pathoram}, eviction takes place by writing back the same path that was read and placing the remapped block 
along the path at a node that intersects with the path to the new leaf to which the block is remapped. Further, PathORAM \cite{pathoram} uses constant 
sized blocks in the presence of a logarithmic-sized client side {\em stash} to handle overflows.

{\bf RingORAM:} RingORAM decouples fetching a path during an access and eviction, by evicting along a deterministically chosen path after a fixed number of fetches. 
Similar to \cite{GentryORAM}, the path is chosen in the reverse-lexicographical order for better eviction. Further, RingORAM uses larger buckets but reads 
only one block per bucket during a query as determined by a per-bucket metadata.

\subsection{Recent Developments}
\label{oram:related:recent}
%
Recent work \cite{curious} has shown that for practical deployment on clouds, ORAMs such as ObliviStore \cite{oblivistore} and CURIOUS \cite{curious} that 
divide the data into constant sized sub-ORAMs perform better than both pyramid ORAMs or tree based ORAMs. CURIOUS uses PathORAM as the primitive for the 
sub-ORAM construction. 
Alternative ORAM constructions \cite{onionoram,comporam} achieve $\order{1}$ communication complexity at the cost of server side computation using 
homomorphic encryption. However, due to expensive homomorphic computations, these ORAMs are not yet practical for deployment.